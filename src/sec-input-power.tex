
%-------------------------------------------------------------------------
% Design Project Input/Output Module Description
%-------------------------------------------------------------------------

\clearpage
\section{emonTx Power Input Module}
\label{sec-input-power}

This input module enables your IoT device to read the AC voltage and current
used to power lights, appliances, and everything else in your home.  The emonTx
Shield you will use was developed by the Open Energy Monitoring group out of the
UK and is designed to ``create a fully open-source energy monitoring and control
system that is suitable for domestic and industrial application.''  

The emonTx will allow you to interface your Arduino's Analog-to-Digital
Converters (ADCs) to current and voltage sensors.  This enables you to safely
measure the voltage and current that a lamp, for example, uses when plugged into
the wall.  You will be using an AC-to-AC adapter to measure voltage. Using a
transformer, the AC-to-AC adapter reduces the wall voltage down by a factor of
about 12, so 110V becomes 9V.  If the wall voltage changes, you will be able to
see that change, reduced by a factor of about 12.  The current sensor is also a
transformer.  By wrapping a ferrite loop around a wire, you can detect the
induced magnetic field from an AC current in the wire.  This induced field
creates a voltage proportional to the current, which we can measure using the
Arduino.  Power is current times voltage, or $P=IV$, so armed with these two
measurements we can calculate the power in watts consumed by a household device.

A sample circuit and Arduino code is shown below to get you started.  
The emonTx plugs directly into the Arduino, the current sensors plug into the
3.5mm audio jacks, and the AC-to-AC connector goes into the barrel connector
jack.  We will share the voltage sensor across all of the current sensors (note
that current sensor 4 is disabled, so don't use it).  The sample code below will
let you calculate power for the devices connected to current sensor 1.

\vspace{0.1in}
\begin{minipage}[t]{0.49\tw}

  \vspace{0.1in}
    Picture goes here!

\end{minipage}
\hfill
\begin{minipage}[t]{0.49\tw}
  \vspace{0.1in}
  \begin{Verbatim}[gobble=3,fontsize=\small]
    // EmonLibrary examples openenergymonitor.org, Licence GNU GPL V3

    #include "EmonLib.h"             // Include Emon Library
    EnergyMonitor emon1;             // Create an instance

    float power_offset = 3.9;        // Include an offset to "zero" the power


    void setup()
    {  
      Serial.begin(9600);
      
      emon1.voltage(0, 120.0, 1.7);  // Voltage: input pin, calibration, phase_shift
      emon1.current(1, 60.6);       // Current: input pin, calibration.
    }

    void loop()
    {
      emon1.calcVI(20,2000);         // Calculate all. No.of half wavelengths
    (crossings), time-out
      float realPower       = emon1.realPower - power_offset;        //extract Real
    Power into variable
      float supplyVoltage   = emon1.Vrms;             //extract Vrms into Variable
      
      // Floor the power at 0 
      if (realPower < 0.0)
        realPower = 0.0;

      Serial.print(realPower); Serial.print("W @ "); 
      Serial.print(emon1.Vrms); Serial.println("V");
    }
  \end{Verbatim}
\end{minipage}
\vspace{0.1in}

%Questions:
