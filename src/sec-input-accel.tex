%-------------------------------------------------------------------------
% Design Project Input/Output Module Description
%-------------------------------------------------------------------------

\clearpage
\section{Accelerometer Input Module}
\label{sec-input-accel}

This input module enables your IoT device to sense physical acceleration
-- perfect for applications requiring tilt-sensing, motion, shock,
vibration, etc. The ADXL335 3-axis accelerometer you will use is a
micro-electro-mechanical system (MEMS) device: a device that combines
tightly-coupled electrical and mechanical systems on a very small scale
(micro-meters!). The accelerometer works by measuring the capacitance
between fixed plates and a tiny structure suspended in mid-air by tiny
springs. As the device experiences acceleration in the X, Y, and/or Z
axes, the suspended structure deflects, changing the capacitance along
each axis. Each capacitance converts to an output voltage, proportional
to the acceleration on that axis, which can then be sensed by the
Arduino.

A sample circuit and Arduino code is shown below to get you started.
The example code will print calibrated accelerometer readings for each
axis on the serial monitor, similar to how we printed the analog reading
from the grayscale sensor in Lab~2. After setting up the circuit and
programming the Arduino, open the serial monitor, place your device on
the table, and check the values the accelerometer is sensing. These
should be close to (x = 0, y = 0, z = 0). Then pick up the device and
tilt it or shake it. See how the values change along the three axes!

\vspace{0.1in}
\begin{minipage}[t]{0.49\tw}

  \vspace{0.0in}
  \includegraphics[width=\tw]{input-accel-annotated.svg.pdf}

  \vspace{0.0in}
  \begin{Verbatim}[gobble=3,fontsize=\small]
    int pin_x_accel = A0;
    int pin_y_accel = A1;
    int pin_z_accel = A2;

    // Calibration

    int xmin = 267; int xmax = 399;
    int ymin = 272; int ymax = 404;
    int zmin = 274; int zmax = 412;

    void setup() {
      Serial.begin(9600);
    }

    // Return average of 10 readings over 10ms
    int ReadAxis( int axisPin ) {
      long reading = 0;
      delay(1);
      for(int i = 0; i < 10; ++i)
        reading += analogRead( axisPin );
      return reading / 10;
    }
  \end{Verbatim}
\end{minipage}
\hfill
\begin{minipage}[t]{0.49\tw}
  \vspace{0.0in}
  \begin{Verbatim}[gobble=3,fontsize=\small]
    void loop() {
      // Read each axis.

      int x_raw = ReadAxis( pin_x_accel );
      int y_raw = ReadAxis( pin_y_accel );
      int z_raw = ReadAxis( pin_z_accel );

      // Scale measurements to calibrated minimum
      // and maximum.

      int x_scaled =
        map( x_raw, x_min, x_max, -1000, 1000 );
      int y_scaled =
        map( y_raw, y_min, y_max, -1000, 1000 );
      int z_scaled =
        map( z_raw, z_min, z_max, -1000, 1000 );

      // Report scaled measurements.

      Serial.print("x: ");
      Serial.print( x_scaled / 1000.0 );
      Serial.print("y: ");
      Serial.print( y_scaled / 1000.0 );
      Serial.print("z: ");
      Serial.print( z_scaled / 1000.0 );
      Serial.println();

      // Wait 500ms before next reading.

      delay(500);
    }
  \end{Verbatim}
\end{minipage}
\vspace{0.1in}

%Questions:
